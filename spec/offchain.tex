\section{Off-Chain Protocol}\label{sec:offchain}

This section describes the actual Coordinated Hydra Head protocol, an even more
simplified version of the original publication~\cite{hydrahead20}. See the
protocol overview in Section~\ref{sec:overview} for an introduction and notable
changes to the original protocol. While the on-chain part already describes the
full life-cycle of a Hydra head on-chain, this section completes the picture by
defining how the protocol behaves off-chain and notably the relationship between
on- and off-chain semantics. Participants of the protocol are also called Hydra
head members, parties or simply protocol actors. The protocol is specified as a
reactive system that processes three kinds of inputs:
\begin{enumerate}
	\item On-chain protocol transactions as introduced in
	      Section~\ref{sec:on-chain}, which are posted to the mainchain and can be
	      observed by all actors:
				\begin{itemize}
					\item $\mathtt{initialTx}$: initiates a head
					\item $\mathtt{commitTx}$: commits UTxO to an initializing head
					\item $\mathtt{collectComTx}$: opens a head
					\red{\item $\mathtt{decrementTx}$: removes UTxO from an open head}
					\item $\mathtt{closeTx}$: closes a head
					\item $\mathtt{contestTx}$: contests a closed head
					% NOTE: fanout not mentioned because not needed in off-chain protocol
					% description
				\end{itemize}
	\item Off-chain network messages sent between protocol actors (parties):
	      \begin{itemize}
		      \item $\hpRT$: to request a transaction to be included in the next snapshot
					\red{\item $\hpRD$: to request exclusion of UTxO via a decommit transaction}
		      \item $\hpRS$: to request a snapshot to be created \& signed by every head member
		      \item $\hpAS$: to acknowledge a snapshot by replying with their signatures
	      \end{itemize}
	\item Commands issued by the participants themselves or on behalf of end-users and clients
	      \begin{itemize}
		      \item $\hpInit$: to start initialization of a head
		      \item $\hpClose$: to request closure of an open head
	      \end{itemize}
\end{enumerate}

% TODO: define states and e.g. that newTx not possible when closed? state diagram?

The behavior is fully specified in Figure~\ref{fig:off-chain-prot}, while the
following paragraphs introduce notation, explain variables and walk-through the
protocol flow.

\subsection{Assumptions}

On top of the statements of the protocol setup in Section~\ref{sec:setup}, the
off-chain protocol logic relies on these assumptions:
\todo{move/merge with protocol setup?}
\begin{itemize}
	\item Every network message received from a specific party is checked for
	      authentication. An implementation of the specification needs to find a
	      suitable means of authentication, either on the communication channel
	      or for individual messages. Unauthenticated messages must be dropped.
	\item The head protocol gets correctly (and with completeness) notified about
	      observed transactions on-chain belonging to the respective head
	      instance.
	      % TODO: Mention multiple heads?
	      % \item The specification covers only a single instance of a Hydra head.
	      %       However, some implementations may choose to track multiple instances. As
	      %       multiple Hydra heads might exist on the same blockchain, it is vital
	      %       that they do not interfere and the specification will take special care
	      %       to ensure this.
	\item All inputs are processed to completion, i.e.\ run-to-completion semantics
	      and no preemption.
	\item Inputs are deduplicated. That is, any two identical inputs must not lead
	      to multiple invocations of the handling semantics.
	\item Given the specification, inputs may pile up forever and implementations
	      need to consider these situations (i.e.\ potential for DoS). A valid reaction
	      to this would be to just drop these inputs. Note that, from a security standpoint,
	      these situations are identical to a non-collaborative peer and closing the head
	      is also a possible reaction.
	\item The lifecycle of a Hydra head on-chain does not cross (hard fork)
	      protocol update boundaries. Note that these inputs are announced in
	      advance hence it should be possible for implementations to react in such
	      a way as to expedite closing of the head before such a protocol update.
	      This further assumes that the contestation period parameter is picked
	      accordingly.\todo{Treat this also in a dedicated section like rollbacks}
\end{itemize}

\subsection{Notation}
\todo{missing:, apply tx}
\begin{itemize}
	\item $\KwOn~event$ specifies how the protocol reacts on a given input $event$.
	      Further information may be available from the constituents of $event$
	      and origin of the input.
	\item $\Req~p$ means that boolean expression $p \in \tyBool$ must be satisfied
	      for the further execution of a routine, while discontinued on $\neg p$.
	      A conservative protocol actor could interpret this as a reason to close
	      the head.
	\item $\KwWait~p$ is a non-blocking wait for boolean predicate $p \in \tyBool$
	      to be satisfied. On $\neg p$, the execution of the routine is stopped,
	      queued, and reactivated at latest when $p$ is satisfied.
	\item $\Multi{}~msg$ means that a message $msg$ is (channel-) authenticated
	      and sent to all participants of this head, including the sender.
	\item $\PostTx{}~tx$ has a party create transaction $tx$, potentially from
	      some data, and submit it on-chain. See Section~\ref{sec:on-chain} for
	      individual transaction details.
	\item $\Out{}~event$ signals an observation of $event$, which is used in the
	      security definition and proofs of Section~\ref{sec:security}. This
	      keyword can be ignored when implementing the protocol.
\end{itemize}

\subsection{Variables}

Besides parameters agreed in the protocol setup (see Section~\ref{sec:setup}), a
party's local state consists of the following variables:

\begin{itemize}
	\item \red{$\hatv$: Last seen open state version.}
	\item $\hats$: Sequence number of latest seen snapshot.
	\item $\hatSigma \in {(\tyNatural \times \tyBytes)}^{*}$: Accumulator of
	      signatures of the latest seen snapshot, indexed by parties.
	\item $\hatmL$: UTxO set representing the local ledger state resulting from
	      applying $\hatmT$ to $\bar{S}.U$ to validate requested transactions.
	\item $\hatmT \in \mT^{*}$: List of transactions applied locally and pending
	      inclusion in a snapshot (if this party is the next leader).
	\item \red{$\tx_{\omega} \in \mathcal{T}^{?}$: Pending decommit transaction, whose outputs are to be decommitted from the head. May be $\bot$ if nothing to decommit.}
	\item \red{$\bar{\mc S}$: Snapshot object of the latest confirmed snapshot which contains:
		\begin{center}
		\begin{tabular}{|l|l|}\hline
			$\bar{\mc S}.v$       & snapshot version \\ \hline
			$\bar{\mc S}.s$       & snapshot number \\ \hline
			$\bar{\mc S}.T$       & set of transactions relating this snapshot to its predecessor \\ \hline
			$\bar{\mc S}.U$       & snapshotted UTxO set \\ \hline
			$\bar{\mc S}.\tx_{\omega}$ & optional decommit transaction \\ \hline
			$\bar{\mc S}.\sigma$ & multisignature \\ \hline
		\end{tabular}
		\end{center}
		The function $\Sno(v,n,T,U,\tx_{\omega})$ initializes a snapshot object with no multi-signature set.
	 \todo{handwavy; types?}
	}
\end{itemize}

\subsection{Protocol flow}

\subsubsection{Initializing the head}

\dparagraph{$\hpInit$.}\quad Before a head can be initialized, all parties need
to exchange and agree on protocol parameters during the protocol setup phase
(see Section~\ref{sec:setup}), so we can assume the public Cardano keys
$\cardanoKeys^{setup}$, Hydra keys $\hydraKeysAgg^{setup}$, as well as the
contestation period $\cPer^{setup}$ are available. One of the clients then can
start head initialization using the $\hpInit$ command, which will result in an
$\mtxInit$ transaction being posted. \\

\dparagraph{$\mathtt{initialTx}$.}\quad All parties will receive this $\mtxInit$
transaction and validate announced parameters against the pre-agreed $setup$
parameters, as well as the structure of the transaction and the minting policy
used. This is a vital step to ensure the initialized Head is valid, which
cannot be checked completely on-chain (see also Section~\ref{sec:init-tx}). \\

\dparagraph{$\mathtt{commitTx}$.}\quad As each party $p_{j}$ posts a
$\mtxCommit$ transaction, the protocol records observed committed UTxOs of each
party $C_j$. With all committed UTxOs known, the $\eta$-state is created (as
defined in Section~\ref{sec:collect-tx}) and the $\mtxCollect$ transaction is
posted. Note that while each participant may post this transaction, only one of
them will be included in the blockchain as the mainchain ledger prevents double
spending. Should any party want to abort, they would post an $\mtxAbort$
transaction and the protocol would end at this point.\\

\dparagraph{$\mathtt{collectComTx}$.}\quad Upon observing the $\mtxCollect$
transaction, the parties compute $\Uinit \gets \bigcup_{j=1}^{n} C_j$ using previously
observed $C_j$ and initialize $\hatmL = \Uinit$. The seen transaction set is
initialized empty $\hatmT = \emptyset$, \red{seen head state version $\hatv = 0$}, as
well as snapshot number $\hats = 0$. \red{No commit transaction is pending
$\tx_{\omega} = \bot$ and} the initial snapshot object is defined accordingly
$\bar{\mc S} \gets \Sno(0, 0, \Uinit, \emptyset, \red{\bot})$.

\subsubsection{Processing transactions off-chain}

Transactions are announced and captured in so-called snapshots. Parties generate
snapshots in a strictly sequential round-robin manner. The party responsible for
issuing the $\ith i$ snapshot is the \emph{leader} of the $\ith i$ snapshot.
Leader selection is round-robin per the $\hydraKeys$ from the protocol setup.
While the frequency of snapshots in the general Head protocol~\cite{hydrahead20}
was configurable, the Coordinated Head protocol does specify a snapshot to be
created after each transaction.\\

\dparagraph{$\hpRT$.}\quad Upon receiving request $(\hpRT,\tx)$, the transaction is
applied to the \emph{local} ledger state $\hatmL \applytx \tx$. If not
applicable yet, the protocol does $\KwWait$ to retry later or eventually marks
this transaction as invalid (see assumption about events piling up). After
applying and if there is no current snapshot in flight ($\hats = \bar{\mc S}.s$) and the
receiving party $\party_{i}$ is the next snapshot
leader, a message to request snapshot signatures $\hpRS$ is sent. \\

\red{\dparagraph{$\hpRD$.}\quad Upon receiving request $(\hpRD,\tx)$, the
	transaction is checked against the \emph{local} ledger state. If decommit is
	not applicable yet, the protocol does $\KwWait$ to retry later and eventually
	emmits an error message that another decommit is in flight. In case there is
	no decommit in flight and party is the leader then $\hpRS$ is
	emmitted containing the decommit transaction $\tx_{\omega}$.} \\

\red{\dparagraph{$\mathtt{decrementTx}$.}\quad Upon observing the \mtxDecrement{}
	transaction, which removed outputs $U_{\omega}$ from the head, the corresponding
	decommit transaction is cleared and the observed version is used for future
	snapshots by
	setting $\hatv = v$. This is required as that the version of the open head state is incremented on each \mtxDecrement{} transaction as described in Section~\ref{sec:decrement-tx}} \\

\dparagraph{$\hpRS$.}\quad Upon receiving request
$(\hpRS,\red{v,}s,\mT_{\mathsf{req}}, \red{\tx_{\omega}})$\footnote{Snapshot requests
	with only transaction identifiers are possible too if all parties keep an
	index of previously seen transactions and their identifiers.} from party
$\party_j$, the receiver $\party_i$ checks that \red{$v$ refers to the current
	open state version}\todo{wait or require?}, $s$ is the next snapshot number
and that party $\party_j$ is responsible for leading its creation.\todo{define
	$\hpLdr$} Party $\party_i$ has to $\KwWait$ until the previous snapshot is
confirmed ($\bar{\mc S}.s = \hats$). \red{If the decommit transaction $\tx_{\omega}$
	is not $\bot$, the transaction is $\Req$d to be applicable to the last confirmed
	UTxO set $\bar{\mc S}.U$ and spent outputs must be removed, yielding the still
	active UTxO set $U_{\mathsf{active}}$. Then, all requested transactions
	$\mT_{\mathsf{req}}$ are $\Req$d to be applicable to $U_{\mathsf{active}}$,}
otherwise the snapshot is rejected as invalid. Only then, $\party_i$ increments
their seen-snapshot counter $\hats$, resets the signature accumulator
$\hatSigma$, and computes the UTxO set of the new local snapshot as
$U \gets \red{U_{\mathsf{active}}} \applytx \mT_{\mathsf{req}}$. Then, $\party_i$
creates a signature $\msSig_i$ using their signing key $\hydraSigningKey$ on a
message comprised by the $\cid$, the new snapshot number $\hats$, the new $\eta$
resulting from canonically combining $U$ (see Section~\ref{sec:close-tx} for
details), \red{and $\eta_{\omega}$ derived from commit transaction $\tx_{\omega}$ outputs}.
The signature is sent to all head members via message $(\hpAS,\hats,\msSig_i)$.
Finally, the local ledger state $\hatmL$ and pending transaction set $\hatmT$
get pruned by re-applying all locally pending transactions $\hatmT$ to the just
requested snapshot's UTxO set iteratively and
ultimately yielding a ``pruned'' version of $\hatmT$ and $\hatmL$. \\

\dparagraph{$\hpAS$.}\quad Upon receiving acknowledgment $(\hpAS,s,\msSig_j)$, all
participants $\Req$ that it is from an expected snapshot (either the last seen
$\hats$ or + 1), potentially $\KwWait$ for the corresponding $\hpRS$ such that
$\hats = s$ and $\Req$ that the signature is not yet included in $\hatSigma$.
They store the received signature in the signature accumulator $\hatSigma$, and
if the signature from each party has been collected, $\party_i$ aggregates the
multisignature $\msCSig$ and $\Req$ it to be valid \red{(using the same message
	as in $\hpRS$)}. If everything is fine, the snapshot can be considered
confirmed by creating the snapshot object
$\bar{\mc S} \gets \Sno(\hatv, \hats, \hatmU, \hatmT \red{, \tx_{\omega}})$ and storing
the multi-signature $\msCSig$ in it for later reference. \red{In case there is a pending
	decommit, any participant can now submit a \mtxDecrement{} transaction by
	providing the just confirmed snapshot with its digests of the active UTxO set
	$\eta$ and the to be removed UTxO set $\eta_{\omega}$.} Similar to the $\hpRT$, if
$\party_i$ is the next snapshot leader and there are already transactions to
snapshot in $\hatmT$, a corresponding $\hpRS$ is distributed.

\subsubsection{Closing the head}

\dparagraph{$\hpClose$.}\quad In order to close a head, a client issues the
$\hpClose$ input which uses the latest confirmed snapshot $\bar{\mc S}$ to
create the new $\eta$-state from the last confirmed UTxO set, the digest of
to-decommit UTxO set $\eta_{\omega}$, and the certifiate $\xi$ using the corresponding
multi-signature. With these, the $\mtxClose$ transaction can be constructed and
posted. See Section~\ref{sec:close-tx} for details about this transaction. \\

\dparagraph{$\mathtt{closeTx}/\mathtt{contestTx}$.}\quad When a party observes
the head getting closed or contested, the $\eta$-state extracted from the
\mtxClose{} or \mtxContest{} transaction represents the latest head status that
has been aggregated on-chain so far (by a sequence of \mtxClose{} and
\mtxContest{} transactions). If the last confirmed (off-chain) snapshot is newer
than the observed (on-chain) snapshot number $s_{c}$, an updated $\eta$-state
and certificate $\xi$ is constructed posted in a \mtxContest{} transaction (see
Section~\ref{sec:contest-tx}).

\subsection{Rollbacks and protocol changes}\label{sec:rollbacks}
The overall life-cycle of the Head protocol is driven by on-chain inputs (see
introduction of Section~\ref{sec:offchain}) which stem from observing
transactions on the mainchain. Most blockchains, however, do only provide
\emph{eventual} consistency. The consensus algorithm ensures a consistent view
of the history of blocks and transactions between all parties, but this
so-called \emph{finality} is only achieved after some time and the local view of
the blockchain history may change until that point.

On Cardano with it's Ouroboros consensus algorithm, this means that any local
view of the mainchain may not be the longest chain and a node may switch to a
longer chain, onto another fork. This other version of the history may not
include what was previously observed and hence, any tracking state needs to be
updated to this ``new reality''. Practically, this means that an observer of the
blockchain sees a \emph{rollback} followed by rollforwards.
% TODO: mention the trade-off about waiting for finality when opening the head
% vs. issue and mark transactions as confirmed on the L2 if they were not in case
% the head opening get's rolled back and not retransmitted.

For the Head protocol, this means that chain events like $\mathtt{closeTx}$ may
be observed a second time. Hence, it is crucial, that the local state of the
Hydra protocol is kept in sync and also rolled back accordingly to be able to
observe and react to these events the right way, e.g.\ correctly contesting this
$\mathtt{closeTx}$ if need be.
% TODO: mention that contestation deadline will stay the same and hence the
% contestation period will need to be picked long enough to reduce the risk of
% not being able to contest anymore after a rollback.

The rollback handling can be specified fully orthogonal on top of the nominal
protocol behavior, if the chain provides strictly monotonically increasing
points $p$ on each chain event via a new or wrapped $\mathtt{rollforward}$
event and $\mathtt{rollback}$ event with the point to which a rollback happened:\\

\dparagraph{$\mathtt{rollforward}$.}\quad On every chain event that is paired or
wrapped in a rollforward event $(\mathtt{rollback},p)$ with point $p$, protocol
participants store their head state indexed by this point in a history
$\Omega$ of states
$\Delta \gets (\hats, \bar{\mc S}.s, \bar{\mc S}.sigma, \hatmU, \barmU, \hatSigma, \hatmL, \hatmT)$ and $\Omega' = (p, \Delta) \cup \Omega$. \\

\dparagraph{$\mathtt{rollback}$.}\quad On a rollback
$(\mathtt{rollback},p_{rb})$ to point $p_{rb}$, the corresponding head state
$\Delta$ need to be retrieved from $\Omega$, with the maximal point
$p \leq p_{rb}$, and all entries in $\Omega$ with $p > p_{rb}$ get removed. \\

This will essentially reset the local head state to the right point and allow
the protocol to progress through the life-cycle normally. Most stages of the
life-cycle are unproblematic if they are rolled back, as long as the protocol
logic behaves as in the nominal case.

A rollback ``past open'' is a special situation though. When a Head is open and
snapshots have been signed, but then a $\mtxCollect$ and one or more
$\mtxCommit$ transactions were rolled back, a bad actor could choose to commit a
different UTxO and open the Head with a different initial UTxO set, while the
already signed snapshots would still be (cryptographically) valid. To mitigate
this, all signatures on snapshots need to incorporate the initial UTxO set by
including $\eta_{0}$. \todo{version-counting is less powerfull}
% FIXME: \eta_0 not compatible with versioning scheme
\todo{Expand to rollbacks in presence of decrements}

\todo{Write about contestation deadline vs. rollbacks}

\begin{figure*}[t!]

	\def\sfact{0.8}
	\centering
	\begin{algobox}{Coordinated Hydra Head}
		\medskip
		\begin{tabular}{c}
			%%% Initializing the head
			\begin{tabular}{cc}
				\adjustbox{valign=t,scale=\sfact}{
					\begin{walgo}{0.6}
						%%% INIT
						\On{$(\hpInit)$ from client}{ %
							$n \gets |\hydraKeys^{setup}|$ \;
							$\hydraKeysAgg \gets \msCombVK(\hydraKeys^{setup})$ \; %
							$\cardanoKeys \gets \cardanoKeys^{setup}$\; %
							$\cPer \gets \cPer^{setup}$ \; %
							$\PostTx{}~(\mtxInit, \nop, \hydraKeysAgg,\cardanoKeys,\cPer)$ \; %
						}
						\vspace{12pt}

						\On{$(\gcChainInitial, \cid, \seed, \nop, \hydraKeysAgg, \cardanoKeys^{\#}, \cPer)$ from chain}{ %
						\Req{} $\hydraKeysAgg=\msCombVK(\hydraKeys^{setup})$\; %
						\Req{} $\cardanoKeys^{\#}= [ \hash(k)~|~\forall k \in \cardanoKeys^{setup}]$\; %
						\Req{} $\cPer=\cPer^{setup}$\; %
						\Req{} $\cid = \hash(\muHead(\seed))$ \; %
						}
					\end{walgo}
				}
				 &

				\adjustbox{valign=t,scale=\sfact}{
					\begin{walgo}{0.6}
						\On{$(\gcChainCommit, j, U)$ from chain}{ %
							$C_j \gets U $

							\If{$\forall k \in [1..n]: C_k \neq \undefined$}{ %
								$\eta \gets (0, \combine([C_1 \dots C_n]))$ \; %
								$\PostTx{}~(\mtxCCom, \eta)$ \; %
							} %
						}

						\vspace{12pt}

						\On{$(\gcChainCollectCom, \eta_{0})$ from chain}{ %
							% Implictly means that all commits will defined as we cannot miss a commit (by assumption)
							$\Uinit \gets \bigcup_{j=1}^{n} U_j$ \; %
							% $\Out~(\hpSnap,(0,U_0))$ \; %
							$\hatmU, \barmU, \hatmL \gets \Uinit$ \; %
							$\hats,\bars \gets 0$ \; %
							$\mT, \hatmT, \barmT \gets \emptyset$ \;
						}

					\end{walgo}
				}
			\end{tabular}

			\\
			\multicolumn{1}{l}{\line(1,0){490}} %
			\\

			%%% Open head
			\begin{tabular}{c@{}c}
				\adjustbox{valign=t,scale=\sfact}{
					\begin{walgo}{0.65}

						%%% NEW TX
						\On{$(\hpNew,\tx)$ from client}{%
							\Multi{} $(\hpRT,\tx)$%
						}

						\vspace{12pt}

						%%% REQ TX
						\On{$(\hpRT,\tx)$ from $\party_j$}{%
							\Wait{$\hatmL \applytx \tx \neq \bot$}{
								$\hatmT \gets \hatmT \cup \{\tx\}$ % candidates for next snapshot

								$\hatmL \gets \hatmL\applytx\tx$ %

								% issue snapshot if we are leader
								\If{$\hats = \bars \land \hpLdr(\bars + 1) = i$}{%
									\Multi{} $(\hpRS,\bars+1,\hatmT)$ \;%
								}
							}
						}

						\vspace{12pt} %

						%%% REQ SN
						\On{$(\hpRS,s,\mT_{res})$ from $\party_j$}{ %

							\Req{$s = \hats + 1 ~ \land ~ \hpLdr(s) = j$} \; %

							% Wait for snapshot no snapshot in flight anymore and all txs applicable
							\Wait{$\bars = \hats ~ \land ~ \barmU \applytx \mT_{res} \not= \bot ~ \and ~ \mT_{res} \subseteq \hatmT$}{ %
								$\hats \gets \bars + 1$ \; %

								$\hatmU \gets \barmU \applytx \mT_{res}$ \; %

								$\eta' \gets (\hats, \combine(\hatmU))$ \; %

								% NOTE: WE could make included transactions auditable by adding
								% a merkle tree root to the (signed) snapshot data \eta'
								$\msSig_i \gets \msSign(\hydraSigningKey, (\cid || \eta_{0} || \eta'))$ \; %
								$\hatSigma \gets \emptyset$

								$\Multi{}~(\hpAS,\hats,\msSig_i)$ \; %

								$\forall \tx \in \mT_{res}: \Out~(\hpSeen,\tx)$ \; %

								% TODO: We would not be keeping all applicable transactions. For
								% this we would need to do a more recursive algorithm. Right now
								% we throw out everything which is not applicable after a single
								% pass. Also: Should we inform users if we drop a transaction?
								\textcolor{red}{
									$X\gets\hatmT$;\ $\hatmT\gets\emptyset$ \; %
									\ForA{$\tx\in X$}{
										\If{$\hatmU\applytx(\hatmT\cup\{\tx\})\not=\bot$}{
											$\hatmT\gets\hatmT\cup\{\tx\}$
										}
									}}
								$\hatmL \gets \hatmU\applytx\hatmT$
								% TODO: This was here ^^^ before:
								% $\hatmT \gets \{ tx \in \hatmT ~  | ~ \hatmU \applytx tx \not = \bot \}$ \; %
								% $\hatmL \gets \hatmU\applytx\hatmT$
							}
						}
					\end{walgo}
				} &

				\adjustbox{valign=t,scale=\sfact}{
					\begin{walgo}{0.6}
						%%% ACK SN
						\On{$(\hpAS,s,\msSig_j)$ from $\party_j$}{ %

							\Req{} $s \in \{\hats,\hats+1\}$
							\; %

							\Wait{$\hats=s$
							}{ %

								\Req{} $(j, \cdot) \notin \hatSigma$ \; %


								\If{$\forall k \in [1..n]: (k,\cdot) \in \hatSigma$}{ %
									% TODO: MS-ASig used different than in the preliminaries
									$\msCSig \gets \msComb(\hydraKeys^{setup}, \hatSigma)$ \; %

									$\eta' \gets (\hats, \combine(\hatmU))$ \; %
									\Req{} $\msVfy(\hydraKeysAgg, (\textcolor{red}{\cid || \eta_{0} ||} \eta'), \msCSig)$ \;
									$\barmU \gets \hatmU$ \; %
									$\bars \gets \hats$ \; %
									$\barsigma \gets \msCSig$ \; %

									%$\Out~(\hpSnap,(\bars,\barmU))$ \;  %
									$\forall \tx \in \mT_{res} : \Out (\hpConf,\tx)$ \; %

									% issue snapshot if we are leader
									\If{$\hpLdr(\bars + 1) = i \land \hatmT \neq \emptyset$}{%
										\Multi{} $(\hpRS,\bars+1,\hatmT)$ \;%
									}
								}
							} }
					\end{walgo}

				}
			\end{tabular}

			\\
			\multicolumn{1}{l}{\line(1,0){490}} %
			\\

			%%% Closing the head
			\begin{tabular}{c c}
				\adjustbox{valign=t,scale=\sfact}{
					\begin{walgo}{0.6}

						% CLOSE from client
						\On{$(\hpClose)$ from client}{ %
							$\eta' \gets (\bars, \combine(\barmU))$ \; %
							$\xi \gets \barsigma$ \; %
							$\PostTx{}~(\mtxClose, \eta', \xi)$ \; %
						}

					\end{walgo}
				}
				 & \adjustbox{valign=t,scale=\sfact}{
					\begin{walgo}{0.6}

						\On{$(\gcChainClose, \eta) \lor (\gcChainContest, \eta)$ from chain}{ %
							$(s_{c}, \cdot) \gets \eta$ \;
							\If{$\bars > s_{c}$}{%
								$\eta' \gets (\bars, \combine(\barmU))$ \; %
								$\xi \gets \barsigma$ \; %
								$\PostTx{}~(\mtxContest, \eta', \xi)$ \; %
							} }

					\end{walgo}
				}
			\end{tabular}
		\end{tabular}
		\bigskip
	\end{algobox}

	\caption{Head-protocol machine for the \emph{coordinated head} from the perspective of party $\party_i$.}\label{fig:off-chain-prot}


\end{figure*}



%%% Local Variables:
%%% mode: latex
%%% TeX-master: "main"
%%% End:


\todo{In figure: $\combine$ on UTxO slightly different than on commits}

%%% Local Variables:
%%% mode: latex
%%% TeX-master: "main"
%%% End:
